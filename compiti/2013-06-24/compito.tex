\documentclass{article}[10pt]
\usepackage[pdftex]{graphicx}
\usepackage{amsfonts}
\usepackage[italian]{babel}
%****************enlarge layout
\textheight     243.5mm
\topmargin      -20.0mm
\textwidth      480pt
\hoffset        -80pt
%*****************theorems and such
\newcounter{esnu}
\newenvironment{esercizio}{\medskip \noindent {\bf Esercizio\addtocounter{esnu}{1} \arabic{esnu}}}{}
\pagestyle{empty}
\newcommand{\liff}{\mathrel{\leftrightarrow}}   % Logical IFF Symbol
\newcommand{\metaiff}{\Longleftrightarrow}      %iff in metatheory

\begin{document}

\begin{center} {\bf Esame di Programmazione I, 24 giugno 2013. 2 ore}\end{center}

\begin{esercizio}
\textbf{[13 punti]}
Si scriva una funzione

\begin{verbatim}
char *somma_modulo_10(const char *s)
\end{verbatim}

\noindent
che restituisce la somma delle cifre contenute in \texttt{s}, modulo $10$. La somma deve essere restituita
come una nuova stringa. Per esempio, un'esecuzione del seguente programma:

{\small
\begin{verbatim}
int main(void) {
  char buffer[101];
  char *s;

  scanf("%100s", buffer);
  printf("somma delle cifre modulo dieci = %s\n", s = somma_modulo_10(buffer));
  free(s)

  return 0;
}
\end{verbatim}
}

\noindent
deve essere la seguente:

{\small
\begin{verbatim}
Ciao2013Oggi67Domani34Hello                   <- input immesso dall'utente
somma delle cifre modulo dieci = sei          <- risposta del calcolatore
\end{verbatim}
}

\end{esercizio}

\begin{esercizio}
\textbf{[9 punti]}
Si considerino le liste di caratteri viste a lezione. Si implementi una funzione \underline{ricorsiva}:
\begin{verbatim}
int alternata(struct list *this)
\end{verbatim}
che determina se la lista \texttt{this} \`e fatta da un'alternanza di singole vocali (minuscole o
maiuscole) e di singoli caratteri non vocali. Se tutto \`e corretto, l'esecuzione del programma:

{\small
\begin{verbatim}
#include <stdio.h>
#include "list.h"

int main(void) {
  struct list *l1 = construct_list
    ('a', construct_list('f', construct_list('I', construct_list('*', construct_list('e', NULL)))));
  struct list *l2 = construct_list('a', construct_list('f', construct_list('&', NULL)));

  printf("l1 = "); print_list(l1); printf("\n");
  if (alternata(l1))
    printf("l1 e' alternata\n");
  else
    printf("l1 non e' alternata\n");

  printf("l2 = "); print_list(l2); printf("\n");
  if (alternata(l2))
    printf("l2 e' alternata\n");
  else
    printf("l2 non e' alternata\n");

  return 0;
}
\end{verbatim}
}
\noindent
dovr\`a stampare:

{\small
\begin{verbatim}
l1 = [a, f, I, *, e]
l1 e' alternata
l2 = [a, f, &]
l2 non e' alternata
\end{verbatim}
}
%
\end{esercizio}

\newpage
\begin{esercizio}
\textbf{[9 punti]}
%
Le note musicali sono normalmente divise in 12 \emph{semitoni}, come nella seguente tabella:

\begin{center}
\begin{tabular}{c|c|c}
  semitono & nome italiano & nome inglese \\\hline
  0 & do & C \\
  1 & do\# & C\#\\
  2 & re & D\\
  3 & re\# & D\#\\
  4 & mi & E \\
  5 & fa & F \\
  6 & fa\# & F\# \\
  7 & sol & G \\
  8 & sol\# & G\# \\
  9 & la & A \\
  10 & la\# & A\# \\
  11 & si & B
\end{tabular}
\end{center}

\noindent
Si scrivano i file \texttt{note.h} e \texttt{nota.c} in modo da definire una struttura \texttt{nota} con le
seguenti funzioni:
%
{\small
\begin{verbatim}
struct nota *construct_nota(int semitono);
void destroy_nota(struct nota *this);
void print_nota_it(struct nota *this);
void print_nota_uk(struct nota *this);
void incrementa_nota(struct nota *this, int inc);
struct nota *nota_incrementata(struct nota *this, int inc);
\end{verbatim}
}

\noindent
dove si assume che \texttt{construct\_nota} riceva sempre un numero tra $0$ e $11$;
\texttt{print\_nota\_it} stampa il nome italiano della nota \texttt{this} mentre
\texttt{print\_nota\_uk} ne stampa il nome inglese; \texttt{incrementa\_nota} incrementa di
\texttt{inc} il semitono della nota \texttt{this} (in modo ciclico: dopo il ``si'' ricominciamo
dal ``do''); \texttt{nota\_incrementata} restituisce una \emph{nuova} nota ottenuta incrementando
di \texttt{inc} il semitono di \texttt{this} (che non viene modificato).

Se tutto \`e corretto, l'esecuzione del seguente programma:

{\small
\begin{verbatim}
#include <stdio.h>
#include "note.h"

int main(void) {
  struct nota *n1 = construct_nota(5); struct nota *n2 = construct_nota(11); struct nota *n3;

  printf("n1: "); print_nota_it(n1); printf(" ");
  print_nota_uk(n1); printf("\n");
  printf("n2: "); print_nota_it(n2); printf(" ");
  print_nota_uk(n2); printf("\n");

  incrementa_nota(n1, 11);
  n3 = nota_incrementata(n2, 23);

  printf("n1: "); print_nota_it(n1); printf(" ");
  print_nota_uk(n1); printf("\n");
  printf("n2: "); print_nota_it(n2); printf(" ");
  print_nota_uk(n2); printf("\n");
  printf("n3: "); print_nota_it(n3); printf(" ");
  print_nota_uk(n3); printf("\n");

  destroy_nota(n1); destroy_nota(n2); destroy_nota(n3);
  return 0;
}
\end{verbatim}}
\noindent
deve stampare:

{\small
\begin{verbatim}
n1: fa F
n2: si B
n1: mi E              <- n1 e' stato modificato
n2: si B              <- n2 non e' stato modificato!
n3: la# A#
\end{verbatim}
}

\end{esercizio}
%
\end{document}
