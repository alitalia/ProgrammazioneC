\documentclass{article}[10pt]
\usepackage[pdftex]{graphicx}
\usepackage{amsfonts}
\usepackage[italian]{babel}
%****************enlarge layout
\textheight     243.5mm
\topmargin      -20.0mm
\textwidth      480pt
\hoffset        -80pt
%*****************theorems and such
\newcounter{esnu}
\newenvironment{esercizio}{\medskip \noindent {\bf Esercizio\addtocounter{esnu}{1} \arabic{esnu}}}{}
\pagestyle{empty}
\newcommand{\liff}{\mathrel{\leftrightarrow}}   % Logical IFF Symbol
\newcommand{\metaiff}{\Longleftrightarrow}      %iff in metatheory

%

%\`e

\begin{document}

\begin{center} {\bf Esame di Programmazione I, 7 febbraio 2014. 2 ore}\end{center}

\begin{esercizio}
\textbf{[12 punti]}
Si scriva una funzione

\begin{verbatim}
int included(const char *where, const char *what)
\end{verbatim}

\noindent
che determina se i caratteri di \texttt{what} si trovano tutti dentro
\texttt{where} e nello stesso ordine. Per esempio, l'esecuzione del seguente programma:

{\small
\begin{verbatim}
#include <stdio.h>

int main(void) {
  const char *s = "Major Tom to Ground Control";

  printf("%i\n", included(s, "jomo"));
  printf("%i\n", included(s, "Tootro"));
  printf("%i\n", included(s, "troll"));
  printf("%i\n", included(s, ""));
  printf("%i\n", included("", "troll"));

  return 0;
}
\end{verbatim}
}

\noindent
dovr\`a stampare:
%
{\small
\begin{verbatim}
1
1
0
1
0
\end{verbatim}
}
\end{esercizio}

\begin{esercizio}
\textbf{[13 punti]}
Considerando le liste di interi come viste a lezione, si scriva una funzione \underline{ricorsiva}
o che chiama una funzione ricorsiva:

\begin{verbatim}
void print_sums(struct list *this)
\end{verbatim}

\noindent
che stampa tutte le possibili somme di 0 o pi\`u elementi della lista \texttt{this}.
Per esempio, il seguente programma:
%
{\small
\begin{verbatim}
#include <stdio.h>
#include "list.h"

int main(void) {
  struct list *l = construct_list(13, construct_list(11, construct_list(-2, construct_list(10, NULL))));

  print_list(l);
  printf("\n");
  print_sums(l);

  return 0;
}
\end{verbatim}
}

\noindent
dovr\`a stampare:
%
{\small
\begin{verbatim}
[13, 11, -2, 10]
 32  22  34  24  21  11  23  13  19   9  21  11   8  -2  10   0 
\end{verbatim}
}
\end{esercizio}

\begin{esercizio}
\textbf{[7 punti]}
Si scrivano i file \texttt{chart.h} e \texttt{chart.c} che definiscono e implementano
una classifica delle dieci canzoni pi\`u ascoltate. Devono essere realizzate le
seguenti funzioni:

{\small
\begin{verbatim}
struct chart *construct_chart();
void destroy_chart(struct chart *this);
void print_chart(struct chart *this);
void set_song_title(struct chart *this, const char *title, int position); 
\end{verbatim}
}

\noindent
L'ultima funzione assegna il titolo della canzone posizionata tra 1 e 10 inclusi.
La terza stampa la classifica, numerando
le canzoni da 1 a 10, usando un trattino per le posizioni per cui non si \`e ancora assegnata la canzone.
Quando una classifica \`e stata appena costruita, nessuna posizione ha ancora una canzone assegnata.

Se tutto \`e corretto, l'esecuzione del programma:

{\small
\begin{verbatim}
int main(void) {
  struct chart *c = construct_chart();

  set_song_title(c, "O luna tua", 3);
  set_song_title(c, "Canzone stonata", 1);
  set_song_title(c, "Red submarine", 8);

  print_chart(c);

  return 0;
}

\end{verbatim}
}
\noindent
dovr\`a stampare:

{\small
\begin{verbatim}
song list
[1] Canzone stonata
[2] -
[3] O luna tua
[4] -
[5] -
[6] -
[7] -
[8] Red submarine
[9] -
[10] -
\end{verbatim}
}
%
\end{esercizio}
%
\end{document}
