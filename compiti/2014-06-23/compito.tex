\documentclass[12pt]{article}
\usepackage{amssymb}
\usepackage{amsmath}
\usepackage{graphicx}

\thispagestyle{empty}
\setlength{\textwidth}{18.5cm}
\setlength{\topmargin}{-2.5cm}
\setlength{\textheight}{24.5cm}
\setlength{\oddsidemargin}{-1cm}
\setlength{\evensidemargin}{-1cm}
\begin{document}
\begin{center}{\LARGE Esame di Programmazione I}\\
\vspace*{-3ex}
\begin{center}
  \large 23 giugno 2014 (tempo disponibile: 2 ore)
\end{center}
\end{center}
%\mbox{}\\
%\vspace*{1ex}
\begin{center}{\Large Esercizio 1} ($13$ punti)
\end{center}
Assumendo che la stringa \texttt{s} contenga solo cifre fra 0 e 9,
si definisca una funzione
%
\begin{verbatim}
char *cifre(const char *s) {
\end{verbatim}
%
che restituisce una nuova stringa contenente la traduzione in italiano
delle cifre di \texttt{s}.

Se tutto \`e corretto, l'esecuzione del seguente programma:

{\scriptsize
\begin{verbatim}
int main(void) {
  char *temp;
  printf("%s\n", temp = cifre("1234")); free(temp);
  printf("%s\n", temp = cifre("034091")); free(temp);
  printf("%s\n", temp = cifre("")); free(temp);
  printf("%s\n", temp = cifre("9")); free(temp); return 0;
}
\end{verbatim}}

\noindent
stamper\`a:
%
{\scriptsize
\begin{verbatim}
uno due tre quattro
zero tre quattro zero nove uno
                   <- qui c'e' una stringa vuota
nove
\end{verbatim}}

\begin{center}{\Large Esercizio 2} ($8$ punti)\end{center}
%
Si considerino le liste di \texttt{char} come viste a lezione. Si definisca una funzione \textbf{ricorsiva}
%
\begin{verbatim}
struct list *reflect(struct list *this)
\end{verbatim}
%
che restituisce una lista ottenuta riflettendo \texttt{this} rispetto
al suo ultimo elemento. La lista \texttt{this} non deve essere modificata.
Se tutto \`e corretto, l'esecuzione del programma

{\scriptsize
\begin{verbatim}
int main(void) {
  struct list *l = construct_list('a', construct_list('b', construct_list('c', NULL)));
  print_list(reflect(l));
  printf("\n");
  return 0;
}
\end{verbatim}}

\noindent
dovr\`a stampare
%
{\scriptsize
\begin{verbatim}
[a, b, c, b, a]
\end{verbatim}}

\begin{center}{\Large Esercizio 3} ($11$ punti)\end{center}
%
Si consideri il seguente file \texttt{vocabolario.h} che definisce la struttura e le funzioni per un vocabolario italiano/inglese
che pu\`o contenere fino a 100 voci:

{\scriptsize
\begin{verbatim}
#ifndef VOCABOLARIO_H
#define VOCABOLARIO_H
#define SIZE 100

struct voce {     // una singola voce italiano/inglese
  char *italiano;
  char *inglese;
};



struct vocabolario {  // il vero e proprio vocabolario
  struct voce voci[SIZE];  // le voci contenute nel vocabolario, non tutte usate
  int voci_usate; // quante voci del vocabolario sono gia' state usate
};

struct vocabolario *construct_vocabolario(); // costruisce un vocabolario vuoto
void destroy_vocabolario(struct vocabolario *this); // dealloca un vocabolario
void print_vocabolario(struct vocabolario *this);  // stampa sul video un vocabolario
void add_in_vocabolario(struct vocabolario *this, char *italiano, char *inglese);  // aggiunge una voce al vocabolario
char *cerca_dallitaliano(struct vocabolario *this, char *italiano);  // cerca una traduzione dall'italiano
char *cerca_dallinglese(struct vocabolario *this, char *inglese);  // cerca una traduzione dall'inglese
void sort_italiano(struct vocabolario *this); // ordina il vocabolario rispetto all'italiano
#endif
\end{verbatim}}

\noindent
Si completi la seguente implementazione \texttt{vocabolario.c}:

{\scriptsize
\begin{verbatim}
#include <stdlib.h>
#include <stdio.h>
#include "vocabolario.h"

void print_vocabolario(struct vocabolario *this) {
  int pos;
  printf("%-20s | %-20s\n", "italiano", "inglese");
  printf("%-20s | %-20s\n", "--------", "-------");
  for (pos = 0; pos < this->voci_usate; pos++)
      printf("%-20s | %-20s\n", this->voci[pos].italiano, this->voci[pos].inglese);
}
....
\end{verbatim}}

\noindent
Se tutto \`e corretto, l'esecuzione del seguente programma:

{\scriptsize
\begin{verbatim}
#include <stdio.h>
#include "vocabolario.h"

int main(void) {
  struct vocabolario *v = construct_vocabolario();
  add_in_vocabolario(v, "cane", "dog"); add_in_vocabolario(v, "topo", "mouse");
  add_in_vocabolario(v, "casa", "house"); add_in_vocabolario(v, "simpatico", "nice");
  add_in_vocabolario(v, "correre", "to run");
  print_vocabolario(v);
  printf("La traduzione in inglese di 'simpatico' e' %s\n", cerca_dallitaliano(v, "simpatico"));
  printf("La traduzione in italiano di 'house' e' %s\n", cerca_dallinglese(v, "house"));
  sort_italiano(v);
  print_vocabolario(v);
  destroy_vocabolario(v); return 0;
}
\end{verbatim}}

\noindent
dovr\`a stampare:

{\scriptsize
\begin{verbatim}
italiano             | inglese             
--------             | -------             
cane                 | dog                 
topo                 | mouse               
casa                 | house               
simpatico            | nice                
correre              | to run              
La traduzione in inglese di 'simpatico' e' nice
La traduzione in italiano di 'house' e' casa
italiano             | inglese             
--------             | -------             
cane                 | dog                 
casa                 | house               
correre              | to run              
simpatico            | nice                
topo                 | mouse
\end{verbatim}}

\end{document}
