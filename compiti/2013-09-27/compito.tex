\documentclass{article}[10pt]
\usepackage[pdftex]{graphicx}
\usepackage{amsfonts}
\usepackage[italian]{babel}
%****************enlarge layout
\textheight     243.5mm
\topmargin      -20.0mm
\textwidth      480pt
\hoffset        -80pt
%*****************theorems and such
\newcounter{esnu}
\newenvironment{esercizio}{\medskip \noindent {\bf Esercizio\addtocounter{esnu}{1} \arabic{esnu}}}{}
\pagestyle{empty}
\newcommand{\liff}{\mathrel{\leftrightarrow}}   % Logical IFF Symbol
\newcommand{\metaiff}{\Longleftrightarrow}      %iff in metatheory

%\`e

\begin{document}

\begin{center} {\bf Esame di Programmazione I, 27 settembre 2013. 2 ore}\end{center}

La rappresentazione \emph{binaria a codice decimale} (bcd) \`e un modo di scrivere in
binario i numeri decimali non negativi, in cui si riservano quattro cifre binarie per ogni cifra
decimale. Per esempio, il numero decimale $209$ viene scritto in bcd come:
\[
  \underbrace{0010}_2\underbrace{0000}_0\underbrace{1001}_9
\]

\begin{esercizio}
\textbf{[12 punti]}
Si scriva una funzione

\begin{verbatim}
char *as_string(int decimal)
\end{verbatim}

\noindent
che restituisce una nuova stringa contenente la rappresentazione bcd del decimale
non negativo passato
come argomento. Si tratta quindi di una stringa di caratteri \texttt{'0'} e
\texttt{'1'}. Per esempio, l'invocazione di \texttt{as\_string(209)} deve restituire la
stringa \verb!"001000001001"!.
L'invocazione di \texttt{as\_string(0)} deve restituire la stringa \verb!"0000"!.
\end{esercizio}

\begin{esercizio}
\textbf{[12 punti]}
Considerando le liste di interi come viste a lezione, si scriva una funzione \underline{ricorsiva}

\begin{verbatim}
struct list *construct_list_bcd(int decimal)
\end{verbatim}

\noindent
che restituisce una nuova lista di 0 e 1 rappresentante il bcd del decimale non negativo
\texttt{decimal}. Per esempio, l'invocazione di \texttt{construct\_list\_bcd(209)} deve
restituire la lista \verb![0, 0, 1, 0, 0, 0, 0, 0, 1, 0, 0, 1]!.
L'invocazione di \texttt{construct\_list\_bcd(0)} deve restituire la
lista \verb![0, 0, 0, 0]!.
\end{esercizio}

\begin{esercizio}
\textbf{[7 punti]}
Si scrivano i file \texttt{bcd.h} e \texttt{bcd.c} che definiscono e implementano
una struttura che rappresenta un numero in bcd. Devono definire e implementare le
seguenti funzioni:

{\small
\begin{verbatim}
struct bcd *construct_bcd(int decimal);
void destroy_bcd(struct bcd *this);
void print_as_decimal(struct bcd *this);
void print_as_binary_string(struct bcd *this);
void print_as_list(struct bcd *this);
\end{verbatim}
}

\noindent
La prima restituisce una nuova struttura che rappresenta in bcd il decimale non negativo
passato come argomento. La seconda dealloca una tale struttura. Le ultime tre stampano a video il
numero bcd rispettivamente come decimale, come stringa binaria e come lista di 0 e 1.

Se tutto \`e corretto, l'esecuzione del programma:

{\small
\begin{verbatim}
#include <stdio.h>
#include "bcd.h"

int main(void) {
  struct bcd *b1 = construct_bcd(209);
  struct bcd *b2 = construct_bcd(0);
  struct bcd *b3 = construct_bcd(1987);

  print_as_decimal(b1); print_as_binary_string(b1); print_as_list(b1); printf("\n");
  print_as_decimal(b2); print_as_binary_string(b2); print_as_list(b2); printf("\n");
  print_as_decimal(b3); print_as_binary_string(b3); print_as_list(b3); printf("\n");

  return 0;
}
\end{verbatim}
}
\noindent
dovr\`a stampare:

{\small
\begin{verbatim}
209
001000001001
[0, 0, 1, 0, 0, 0, 0, 0, 1, 0, 0, 1]

0
0000
[0, 0, 0, 0]

1987
0001100110000111
[0, 0, 0, 1, 1, 0, 0, 1, 1, 0, 0, 0, 0, 1, 1, 1]
\end{verbatim}
}
%
\end{esercizio}
%
\end{document}
