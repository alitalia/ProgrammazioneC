\documentclass{article}[10pt]
\usepackage[pdftex]{graphicx}
\usepackage{amsfonts}
\usepackage[italian]{babel}
%****************enlarge layout
\textheight     243.5mm
\topmargin      -20.0mm
\textwidth      480pt
\hoffset        -80pt
%*****************theorems and such
\newcounter{esnu}
\newenvironment{esercizio}{\medskip \noindent {\bf Esercizio\addtocounter{esnu}{1} \arabic{esnu}}}{}
\pagestyle{empty}
\newcommand{\liff}{\mathrel{\leftrightarrow}}   % Logical IFF Symbol
\newcommand{\metaiff}{\Longleftrightarrow}      %iff in metatheory

\begin{document}

\begin{center} {\bf Esame di Programmazione I, 4 febbraio 2013. 2 ore}\end{center}

\begin{esercizio}
\textbf{[11 punti]}
Si scriva una funzione

\begin{verbatim}
char *mirror(const char *s)
\end{verbatim}

\noindent
che restituisce una nuova stringa ottenuta riflettendo \texttt{s} rispetto al suo ultimo
carattere. Per esempio, l'esecuzione del seguente programma:

{\small
\begin{verbatim}
int main(void) {
  char *buffer;

  printf("%s\n", buffer = mirror("ciao")); free(buffer);
  printf("%s\n", buffer = mirror("che bella giornata!")); free(buffer);
  printf("%s\n", buffer = mirror("")); free(buffer);

  return 0;
}
\end{verbatim}
}

\noindent
deve stampare:

{\small
\begin{verbatim}
ciaoaic
che bella giornata!atanroig alleb ehc
                                          <- qui c'e' la stringa vuota!
\end{verbatim}
}

\end{esercizio}

\begin{esercizio}
\textbf{[10 punti]}
Si considerino le liste viste a lezione. Si implementi una funzione \underline{ricorsiva}:
\begin{verbatim}
struct list *lastToFirst(struct list *this)
\end{verbatim}
che restituisce una lista uguale a \texttt{this} tranne l'ultimo elemento di \texttt{this},
che nella lista risultante si trova spostato all'inizio. La lista \texttt{this} non deve
essere modificata.
Se tutto \`e corretto, l'esecuzione del programma:

{\small
\begin{verbatim}
#include <stdio.h>
#include "list.h"

int main(void) {
  struct list *l1 = construct_list(10, construct_list(-5, construct_list
   (23, construct_list(11, construct_list(-2, NULL)))));

  struct list *l2 = construct_list(10, NULL);
  struct list *l3 = NULL;

  printf("l1 = ");
  print_list(l1);
  printf("\n");

  printf("lastToFirst(l1) = ");
  print_list(lastToFirst(l1));
  printf("\n");

  printf("l1 = ");
  print_list(l1);
  printf("\n");

  printf("lastToFirst(l2) = ");
  print_list(lastToFirst(l2));
  printf("\n");

  printf("lastToFirst(l3) = ");
  print_list(lastToFirst(l3));
  printf("\n");

  return 0;
}
\end{verbatim}
}
\noindent
dovr\`a stampare:

{\small
\begin{verbatim}
l1 = [10, -5, 23, 11, -2]
lastToFirst(l1) = [-2, 10, -5, 23, 11]
l1 = [10, -5, 23, 11, -2]
lastToFirst(l2) = [10]
lastToFirst(l3) = []
\end{verbatim}
}
%
\end{esercizio}

\begin{esercizio}
\textbf{[10 punti]}
%
Si definisca una struttura \texttt{television} che implementa una televisione, sintonizzabile
su un insieme di canali.
Si scrivano i file \texttt{television.h} e \texttt{television.c} implementando le funzioni:
%
\begin{itemize}
\item \texttt{struct television *construct\_television(const char *names[], int num\_channels)} che resti\-tui\-sce una
      nuova televisione con \texttt{num\_channel} canali chiamati come indicato dall'array, che si assume che sia
      di lunghezza \texttt{num\_channel}; la televisione nasce sintonizzata sul canale numero 0;
\item \texttt{void destruct\_television(struct television *this)}, che dealloca la televisione \texttt{this};
\item \texttt{void set\_channel(struct television *this, int channel)}, che sintonizza la televisione sul canale
      numero \texttt{channel}, che si assume tra 0 (incluso) e il numero di canali della televisione (escluso);
\item \texttt{void undo(struct television *this)}, che torna indietro nella storia, cio\`e
      sintonizza la televisione sul canale su cui era sintonizzata
      prima di quello corrente. Se si torna fino all'inizio della storia, questo metodo non fa nulla;
\item \texttt{char *toString(struct television *this)}, che restituisce il nome del canale su cui
      la televisione \texttt{this} \`e in questo momento sintonizzata.
\end{itemize}
%
Se tutto \`e corretto, l'esecuzione del seguente programma:

{\small
\begin{verbatim}
#include <stdio.h>
#include "television.h"

int main(void) {
  const char *names[] = { "raggi zero", "canale uno", "sei sul due" };

  struct television *t = construct_television(names, 3);

                           printf("inizio:            %s\n", toString(t));
  set_channel(t, 2);       printf("set_channel(t, 2): %s\n", toString(t));
  set_channel(t, 1);       printf("set_channel(t, 1): %s\n", toString(t));
  set_channel(t, 2);       printf("set_channel(t, 2): %s\n", toString(t));
  undo(t);                 printf("undo(t):           %s\n", toString(t));
  undo(t);                 printf("undo(t):           %s\n", toString(t));
  undo(t);                 printf("undo(t):           %s\n", toString(t));
  undo(t);                 printf("undo(t):           %s\n", toString(t));
  set_channel(t, 2);       printf("set_channel(t, 2): %s\n", toString(t));
  undo(t);                 printf("undo(t):           %s\n", toString(t));

  destruct_television(t);

  return 0;
}
\end{verbatim}}
\noindent
deve stampare:

{\small
\begin{verbatim}
inizio:            raggi zero
set_channel(t, 2): sei sul due
set_channel(t, 1): canale uno
set_channel(t, 2): sei sul due
undo(t):           canale uno
undo(t):           sei sul due
undo(t):           raggi zero
undo(t):           raggi zero               <- qui la storia e' ormai finita e undo(t) non fa nulla
set_channel(t, 2): sei sul due
undo(t):           raggi zero
\end{verbatim}
}

\end{esercizio}
%
\end{document}
