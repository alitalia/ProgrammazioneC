\documentclass[12pt]{article}
\usepackage{amssymb}
\usepackage{amsmath}
\usepackage{color}
\usepackage{graphicx}

\definecolor{grey}{rgb}{0.3,0.3,0.3}

\usepackage{listings, framed}
\lstset{
  language=Java,
  showstringspaces=false,
  columns=flexible,
  basicstyle={\small\ttfamily},
  frame=none,
  numbers=none,
  keywordstyle=\bfseries\color{grey},
  commentstyle=\itshape\color{red},
  identifierstyle=\color{black},
  stringstyle=\color{blue},
  numberstyle={\ttfamily},
%  breaklines=true,
  breakatwhitespace=true,
  tabsize=3,
  escapechar=|
}

\thispagestyle{empty}
\setlength{\textwidth}{18.5cm}
\setlength{\topmargin}{-2.5cm}
\setlength{\textheight}{24.5cm}
\setlength{\oddsidemargin}{-1cm}
\setlength{\evensidemargin}{-1cm}
\begin{document}
\begin{center}{\LARGE Compito di Programmazione I - BioInformatica}\\
\vspace*{-2ex}
\begin{center}
  \large 21 Giugno 2018 (tempo disponibile: 2 ore)
\end{center}
\end{center}

\begin{center}{\Large Esercizio 1} ($6$ punti)
\end{center}
Che numero stampa l'esecuzione del seguente programma?
\begin{lstlisting}
#include <stdio.h>

int main(void) {
  int original[] = { -8, 6, 7, -5, 8, -1, 0, 4, 3, 1 };
  int counter = 0, i;
  for (i = 0; i < 10; i++) {
    int x = original[i];
    if (x < 0)
      x = -x;

    counter += x % 2;
  }
  printf("%d\n", counter); // cosa stampa?
  return 0;
}
\end{lstlisting}

\vspace*{1ex}
\begin{center}{\Large Esercizio 2} ($10$ punti)
\end{center}
Si scriva una funzione \texttt{max\_array} che riceve due array di interi \texttt{original} e \texttt{result} e la loro lunghezza \texttt{length} (uguale per entrambi). La funzione deve modificare \texttt{result} in modo che ogni suo elemento di indice $i$ diventi il massimo degli elementi di \texttt{original} con indice compreso fra $0$ e $i$ inclusi. La funzione non deve modificare gli elementi di \texttt{original}. Se \texttt{length} \`e minore o uguale a $0$, la funzione non deve fare nulla. Per esempio, se \texttt{original} \`e l'array
\texttt{\{-2, 6, 7, 5, 8, -3, 0, -4, 0, -1\}} e quindi \texttt{length} \`e $10$, alla fine della funzione \texttt{result} deve contenere \texttt{\{-2, 6, 7, 7, 8, 8, 8, 8, 8, 8\}}.


\begin{center}
{\Large Esercizio 3} (12 punti)
\end{center}
Si definisca una struttura \texttt{studente} che implementa uno studente.
Si scrivano i file \texttt{studente.h} e \texttt{studente.c} implementando le funzioni:
%
\begin{itemize}
\itemsep0em
\item \texttt{struct studente *construct\_studente(char *nome)} che restituisce un
      nuovo studente con il nome indicato;
\item \texttt{void destruct\_studente(struct studente *this)} che dealloca lo studente \texttt{this};
\item \texttt{void fa\_esame(struct studente *this, int voto)},
      che registra il voto indicato per lo studente \texttt{this}, se il voto \`e
      fra $18$ e $30$ inclusi, e non fa nulla altrimenti; uno studente pu\`o fare al pi\`u 20 esami:
      oltre tale soglia, questa funzione non registra pi\`u ulteriori esami;
\item \texttt{float media(struct studente *this)}, che restituisce la media dei voti
      degli esami sostenuti dallo studente \texttt{this}; se lo studente non ha ancora fatto esami,
      restituisce $0.0$;
\item \texttt{char *toString(struct studente *this)}, che restituisce una nuova stringa
      fatta dal nome dello studente \texttt{this} seguito dalla media degli esami sostenuti da
      \texttt{this}.
\end{itemize}
%
Se tutto \`e corretto, l'esecuzione del seguente programma:

\begin{lstlisting}
#include <stdlib.h>
#include <stdio.h>
#include "studente.h"

int main(void) {
  struct studente *s1 = construct_studente("Giacomo");
  struct studente *s2 = construct_studente("Elisa");
  char *s;
  fa_esame(s1, 18);
  fa_esame(s1, 15);  // non viene registrato
  fa_esame(s2, 30); fa_esame(s1, 25); fa_esame(s2, 22);
  fa_esame(s2, 29); fa_esame(s2, 27);
  printf("%s\n", s = toString(s1));
  free(s);
  printf("%s\n", s = toString(s2));
  free(s);
  destruct_studente(s1); destruct_studente(s2);
  return 0;
}
\end{lstlisting}

\noindent
deve stampare:

{\small
\begin{verbatim}
Giacomo  21.50
Elisa  27.00
\end{verbatim}
}

\vspace*{1ex}

\begin{center}
{\Large Esercizio 4} (4 punti)
\end{center}

\noindent Cosa stampa il seguente programma C?

\begin{lstlisting}
#include <stdio.h>
 
struct date {
  int day;
  int month;
  int year;
}

void changeDay(struct date d) {
  d.day = 12;  
}
 
int main(void) {
  struct date today;
  today.day = 21;
  today.month = 6;
  today.year = 2018;  
  changeDay(today);
  printf("La data modificata e': %i/%2i/%i", today.day, today.month, today.year);
  return 0;
}
\end{lstlisting}

\end{document}
