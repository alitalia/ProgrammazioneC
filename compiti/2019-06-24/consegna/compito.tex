\documentclass[12pt]{article}
\usepackage{amssymb}
\usepackage{amsmath}
\usepackage{color}
\usepackage{graphicx}

\definecolor{grey}{rgb}{0.3,0.3,0.3}

\usepackage{listings, framed}
\lstset{
  language=Java,
  showstringspaces=false,
  columns=flexible,
  basicstyle={\small\ttfamily},
  frame=none,
  numbers=none,
  keywordstyle=\bfseries\color{grey},
  commentstyle=\itshape\color{red},
  identifierstyle=\color{black},
  stringstyle=\color{blue},
  numberstyle={\ttfamily},
%  breaklines=true,
  breakatwhitespace=true,
  tabsize=3,
  escapechar=|
}

\thispagestyle{empty}
\setlength{\textwidth}{18.5cm}
\setlength{\topmargin}{-2.5cm}
\setlength{\textheight}{24.5cm}
\setlength{\oddsidemargin}{-1cm}
\setlength{\evensidemargin}{-1cm}
\begin{document}
\begin{center}{\LARGE Compito di Programmazione I - BioInformatica}\\
\vspace*{-2ex}
\begin{center}
  \large 24 giugno 2019 (tempo disponibile: 2 ore)
\end{center}
\end{center}

\vspace*{1ex}
\begin{center}{\Large Esercizio 1} ($12$ punti) \textbf{[Si consegni \texttt{orario.c}]}\end{center}

Un istante temporale della giornata viene rappresentato in Italia
con ore, minuti e secondi, separati dal carattere due punti, con le ore che
vanno da 00 a 23 e i minuti e i secondi da 00 a 59. Negli Stati Uniti, invece, si usa una
rappresentazione su dodici ore, da 01 a 12, e si distingue tra prima e seconda
parte della giornata tramite i suffissi AM e PM. Per esempio, alcuni istanti
sono rappresentati in questo modo in Italia e Stati Uniti:

\begin{center}
  \begin{tabular}{|c|c|}
    \hline
    Italia & Stati Uniti \\\hline\hline
    00:00:00 & 12:00:00AM \\\hline
    00:23:08 & 12:23:08AM \\\hline
    01:23:08 & 01:23:08AM \\\hline
    02:23:08 & 02:23:08AM \\\hline
    11:23:08 & 11:23:08AM \\\hline\hline
    12:00:00 & 12:00:00PM \\\hline
    12:23:08 & 12:23:08PM \\\hline
    13:23:08 & 01:23:08PM \\\hline
    14:23:08 & 02:23:08PM \\\hline
    23:23:08 & 11:23:08PM \\\hline
  \end{tabular}
\end{center}
%
Si faccia attenzione in particolare alla rappresentazione della mezzanotte
e del mezzogiorno usata negli Stati Uniti, che pu\`o essere
sorprendente e indurre in errore.

Si scriva un programma \texttt{orario.c} la cui funzione \texttt{main}:

\begin{itemize}
\item legge da tastiera le ore, fra 0 e 23, i minuti, fra 0 e 59, e i secondi, fra 0 e 59. Se non sono
  inseriti correttamente, li richiede ad oltranza;
\item stampa su video l'orario statunitense corrispondente;
\item e infine stampa su video l'orario italiano corrispondente.
\end{itemize}

Per esempio, se venisse inserito 14 (per le ore), 5 (per i minuti) e 12 (per i secondi),
tale programma dovrebbe stampare:

\begin{verbatim}
Stati Uniti: 02:05:12PM
Italia: 14:05:12
\end{verbatim}

Se invece venisse inserito 12 (per le ore), 5 (per i minuti) e 0 (per i secondi),
tale programma dovrebbe stampare:

\begin{verbatim}
Stati Uniti: 12:05:00PM
Italia: 12:05:00
\end{verbatim}

\newpage
\begin{center}{\Large Esercizio 2} ($7$ punti) \textbf{[Si consegni \texttt{spiegazione.txt} ]}\end{center}
Che cosa stampa il seguente programma? Giustificare la risposta.
\begin{lstlisting}
#include <stdio.h>

void cambia(int a, int *b, int * c) {
	a=15;
	*b=7;
	b=c;
}

int main() {
	int x;
	int y;
	int *z, *k;

	x=55;
	y=12;
	k=&x;
	z=&y;

	printf("Valori: x=%d y=%d k=%d z=%d\n", x, y, *k, *z); 
	cambia(y, k, z);
	printf("Valori: x=%d y=%d k=%d z=%d\n", x, y, *k, *z); // cosa stampa?
	return 0;
}
\end{lstlisting}

\begin{center}{\Large Esercizio 3} ($12$ punti) \textbf{[Si consegni \texttt{string.c}]}\end{center}
Una stringa contiene nome e cognome di una persona separati tra loro da uno o pi\`u spazi.
Una seconda stringa contiene il soprannome di una persona, e pu\`o eventualmente contenere
degli spazi. Si scriva una funzione C che prende in ingresso due stringhe del tipo suddetto e restituisce
una nuova stringa in cui tra il nome ed il cognome \`e inserito il soprannome tra parentesi
tonde. Nella nuova stringa, sia tra il nome e la parantesi aperta che tra la parentesi chiusa
ed il cognome deve essere presente uno (ed un solo) spazio.
Ad esempio, se le due stringhe sono \texttt{Bruce Springsteen} e \texttt{The Boss}, la stringa
restituita dalla funzione deve essere: \texttt{Bruce (The Boss) Springsteen}.

\end{document}
