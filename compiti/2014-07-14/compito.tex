\documentclass[12pt]{article}
\usepackage{amssymb}
\usepackage{amsmath}
\usepackage{graphicx}

\thispagestyle{empty}
\setlength{\textwidth}{18.5cm}
\setlength{\topmargin}{-2.5cm}
\setlength{\textheight}{24.5cm}
\setlength{\oddsidemargin}{-1cm}
\setlength{\evensidemargin}{-1cm}
\begin{document}
\begin{center}{\LARGE Esame di Programmazione I}\\
\vspace*{-2ex}
\begin{center}
  \large 14 luglio 2014 (tempo disponibile: 2 ore)
\end{center}
\end{center}
%\mbox{}\\
%\vspace*{1ex}
\begin{center}{\Large Esercizio 1} ($11$ punti)
\end{center}
Si scriva una funzione \texttt{char *camelize(const char *s)}
che restituisce una nuova stringa, che \`e la versione
\emph{camel-style} della stringa \texttt{s}, in cui cio\`e gli spazi
sono stati eliminati e l'inizio delle parole (tranne la prima) \`e
stato scritto in maiuscolo. Si assuma per semplicit\`a che
\texttt{s} contenga solo caratteri alfabetici minuscoli e spazi, non abbia
mai due spazi di seguito e cominci con un carattere alfabetico minuscolo.

Se tutto \`e corretto, l'esecuzione del seguente programma:

{\small
\begin{verbatim}
int main(void) {
  char *s;
  printf("%s\n", s = camelize("camels are sweet and clever animals")); free(s);
  return 0;
}
\end{verbatim}}

\noindent
dovr\`a stampare:
%
{\small
\begin{verbatim}
camelsAreSweetAndCleverAnimals
\end{verbatim}}

\vspace*{1ex}
\begin{center}{\Large Esercizio 2} ($9$ punti)\end{center}
%
Si considerino le liste di \texttt{char} come viste a lezione. Si definisca una funzione
\textbf{ricorsiva}
%
\begin{verbatim}
struct list *camelize_list(struct list *this)
\end{verbatim}
%
che restituisce una lista ottenuta da \texttt{this} traducendola in \emph{camel-style}. Si assuma
per semplicit\`a che la lista \texttt{this} contenga
solo caratteri alfabetici minuscoli e spazi, non abbia
mai due spazi di seguito e cominci con un carattere alfabetico minuscolo.
La lista \texttt{this} non deve venire modificata. Per esempio, l'esecuzione del programma

{\small
\begin{verbatim}
int main(void) {
  struct list *l = construct_list
    ('c', construct_list
     ('a', construct_list
      ('m', construct_list
       ('e', construct_list
        ('l', construct_list
         ('s', construct_list
          (' ', construct_list
           ('a', construct_list
            ('r', construct_list
             ('e', construct_list
              (' ', construct_list
               ('s', construct_list
                ('w', construct_list
                 ('e', construct_list
                  ('e', construct_list
                   ('t', NULL))))))))))))))));
			 
  print_list(camelize_list(l)); printf("\n"); return 0;
}
\end{verbatim}}

\noindent
dovr\`a stampare
%
{\small
\begin{verbatim}
[c, a, m, e, l, s, A, r, e, S, w, e, e, t]
\end{verbatim}}

\vspace*{1ex}
\begin{center}{\Large Esercizio 3} ($12$ punti)\end{center}
%
Si scrivano i file \texttt{date.h} e \texttt{date.c} che definiscono e implementano la struttura
\texttt{date} che rappresenta una data del calendario, con le seguenti funzioni sulle date:

{\small
\begin{verbatim}
struct date *construct_date(int day, int month, int year);
struct date *construct_date_alt(int day, const char *month, int year);
void print_date(struct date *this);  // stampa this, del tipo "13 gennaio 1973"
int equals_date(struct date *this, struct date *that); // vero se e solo se le date sono uguali
int compare_date(struct date *this, struct date *that); // <0 se this viene prima di that,
  // 0 se this e that sono uguali, >0 se this viene dopo that
\end{verbatim}}

\noindent
La funzione di costruzione \texttt{construct\_date\_alt} permette di specificare il mese
come il nome minuscolo italiano del mese. Entramble le funzioni di costruzione
ritornano \texttt{NULL} se si
cerca di costruire una data inesistente e considerano gli anni bisestili.

Se tutto \`e corretto, l'esecuzione del seguente programma:

{\small
\begin{verbatim}
#include <stdio.h>
#include "date.h"

static void compare(struct date *d1, struct date *d2) {
  int comp = compare_date(d1, d2);
  print_date(d1);

  if (comp < 0) printf(" < ");
  else if (comp == 0) printf(" = ");
  else printf(" > ");

  print_date(d2);
  printf("\n");
}

int main(void) {
  struct date *d1 = construct_date(14, 7, 1789);
  struct date *d2 = construct_date_alt(15, "maggio", 1789);
  struct date *d3 = construct_date_alt(15, "giugno", 2014);
  struct date *d4 = construct_date_alt(14, "luglio", 1789);
  compare(d1, d2);
  compare(d2, d3);
  compare(d1, d3);
  compare(d1, d4);
  return 0;
}
\end{verbatim}}

\noindent
dovr\`a stampare:

{\small
\begin{verbatim}
14 luglio 1789 > 15 maggio 1789
15 maggio 1789 < 15 giugno 2014
14 luglio 1789 < 15 giugno 2014
14 luglio 1789 = 14 luglio 1789
\end{verbatim}}

\end{document}
