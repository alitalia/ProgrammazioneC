\documentclass{article}[10pt]
\usepackage[pdftex]{graphicx}
\usepackage{amsfonts}
\usepackage[italian]{babel}
%****************enlarge layout
\textheight     243.5mm
\topmargin      -20.0mm
\textwidth      480pt
\hoffset        -80pt
%*****************theorems and such
\newcounter{esnu}
\newenvironment{esercizio}{\medskip \noindent {\bf Esercizio\addtocounter{esnu}{1} \arabic{esnu}}}{}
\pagestyle{empty}
\newcommand{\liff}{\mathrel{\leftrightarrow}}   % Logical IFF Symbol
\newcommand{\metaiff}{\Longleftrightarrow}      %iff in metatheory

\begin{document}

\begin{tabular}{llclcr}
 \hspace{-35pt} &{\bf COGNOME:} & \hspace{100pt}        &{\bf NOME:}    & \hspace{100pt}        &{\bf MATRICOLA:}\hspace{35pt} \\
\hline
\end{tabular}
\begin{center} {\bf Esame di Programmazione I, 3 settembre 2012. 2 ore}\end{center}

\begin{esercizio}
\textbf{[13 punti]}
Si scriva una funzione

\begin{verbatim}
char *binario(int n)
\end{verbatim}

\noindent
che restituisce una nuova stringa che contiene la rappresentazione binaria del numero \texttt{n}.
Si assuma che sia $\mathtt{n}>0$.

Tale programma deve avere anche una funzione \texttt{main} che:
\begin{enumerate}
\item legge da tastiera un numero $n$ maggiore di $0$; se non \`e maggiore di 0 lo richiede ad oltranza;
\item chiama \texttt{binario} su tale numero e ne stampa la stringa risultante;
\item dealloca la stringa risultante.
\end{enumerate}

Se tutto \`e corretto, un'esecuzione di tale programma dovrebbe essere del tipo:

{\small
\begin{verbatim}
Inserisci un numero positivo: -3
Inserisci un numero positivo: 0
Inserisci un numero positivo: 268
100001100
\end{verbatim}
}

\end{esercizio}

\begin{esercizio}
\textbf{[9 punti]}
Si considerino le liste viste a lezione. Si implementi una funzione ricorsiva:
\begin{verbatim}
struct list *put0(struct list *this);
\end{verbatim}
che restituisce una lista ottenuta da \texttt{this} inserendo, davanti agli elementi di valore dispari, un nuovo nodo che contiene
il valore $0$.

Se tutto \`e corretto, l'esecuzione del programma:

{\small
\begin{verbatim}
#include <stdio.h>
#include <stdlib.h>
#include "list.h"

int main(void) {
  struct list *l =
    construct_list
    (11, construct_list
     (13, construct_list
      (-4, construct_list
       (13, construct_list
        (-5, construct_list
         (0, construct_list
          (1, NULL)))))));

  print_list(l);
  printf("\n");

  print_list(put0(l));
  printf("\n");

  return 0;
}
\end{verbatim}
}
\noindent
dovr\`a stampare:

{\small
\begin{verbatim}
[11, 13, -4, 13, -5, 0, 1]
[0, 11, 0, 13, -4, 0, 13, 0, -5, 0, 0, 1]
\end{verbatim}
}
%
\end{esercizio}
%
\begin{esercizio}
\textbf{[10 punti]}
%
Si definisca una struttura \texttt{banca} che implementa una banca con massimo 10 correntisti,
identificati per nome.
Si scrivano i file \texttt{banca.h} e \texttt{banca.c} implementando le funzioni:
%
\begin{itemize}
\item \texttt{struct banca *construct\_banca()} che restituisce una nuova banca,
      al momento senza correntisti;
\item \texttt{void destruct\_banca(struct banca *this)} che dealloca la banca \texttt{this};
\item \texttt{void deposita(struct banca *this, char *nome, double soldi)},
      che aggiunge i \texttt{soldi} indicati sul conto del correntista chiamato \texttt{nome} che si
      trova nella banca \texttt{this}. Se il correntista non \`e ancora presente
      nella banca e non si
      \`e ancora arrivati al massimo di 10 correntisti, esso viene aggiunto come nuovo correntista con deposito
      iniziale pari a \texttt{soldi}.  Se il correntista non \`e ancora presente
      nella banca e si \`e gi\`a arrivati a 10 correntisti, questa funzione non
      fa nulla;
\item \texttt{void preleva(struct banca *this, char *nome, double soldi)}, che preleva i \texttt{soldi} indicati
      dal conto del correntista \texttt{nome}, riducendo quindi i soldi che \texttt{nome} ha depositati sul conto;
\item \texttt{char *toString(struct banca *this)}, che restituisce una nuova stringa del
      tipo
      \begin{verbatim}
        Fausto: 113.840000
        Samantha: 1023.030000
        Alessandra: 11.340000
      \end{verbatim}
      in cui cio\`e sono riportati i nomi dei correntisti e l'entit\`a del deposito che hanno presso
      la banca.
\end{itemize}
%
Se tutto \`e corretto, l'esecuzione del seguente programma:

{\small
\begin{verbatim}
#include <stdlib.h>
#include <stdio.h>
#include "banca.h"

int main(void) {
  struct banca *b = construct_banca();
  char *s;

  deposita(b, "Fausto", 112.14);
  deposita(b, "Samantha", 1023.03);
  deposita(b, "Fausto", 13.00);
  deposita(b, "Alessandra", 11.34);

  printf("%s\n", s = toString(b));
  free(s);

  preleva(b, "Fausto", 11.30);

  printf("%s\n", s = toString(b));
  free(s);

  destruct_banca(b);
  return 0;
}
\end{verbatim}}

\noindent
deve stampare:

{\small
\begin{verbatim}
Fausto: 125.140000
Samantha: 1023.030000
Alessandra: 11.340000

Fausto: 113.840000
Samantha: 1023.030000
Alessandra: 11.340000
\end{verbatim}}

\end{esercizio}
%
\end{document}
