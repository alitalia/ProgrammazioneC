\documentclass[12pt]{article}
\usepackage{amssymb}
\usepackage{amsmath}
\usepackage{color}
\usepackage{graphicx}

\definecolor{grey}{rgb}{0.3,0.3,0.3}

\usepackage{listings, framed}
\lstset{
  language=Java,
  showstringspaces=false,
  columns=flexible,
  basicstyle={\small\ttfamily},
  frame=none,
  numbers=none,
  keywordstyle=\bfseries\color{grey},
  commentstyle=\itshape\color{red},
  identifierstyle=\color{black},
  stringstyle=\color{blue},
  numberstyle={\ttfamily},
%  breaklines=true,
  breakatwhitespace=true,
  tabsize=3,
  escapechar=|
}

\thispagestyle{empty}
\setlength{\textwidth}{18.5cm}
\setlength{\topmargin}{-2.5cm}
\setlength{\textheight}{24.5cm}
\setlength{\oddsidemargin}{-1cm}
\setlength{\evensidemargin}{-1cm}
\begin{document}
\begin{center}{\LARGE Esame Completo di Programmazione I - BioInformatica}\\
\vspace*{-2ex}
\begin{center}
  \large 1 febbraio 2019 (tempo disponibile: 2 ore)
\end{center}
\end{center}

\vspace*{1ex}
\begin{center}{\Large Esercizio 1} ($6$ punti)
\end{center}
Si scriva un programma \texttt{sum.c} che implementa una funzione \texttt{int sum(int arr[], int length)}. Tale funzione deve ricevere un array \texttt{arr} di interi, lungo \texttt{length}, e deve restituire la somma degli elementi di \texttt{arr} che abbiano l'elemento precedente dispari. Per esempio, se \texttt{arr} fosse $\{2,3,4,1,5\}$, la funzione dovrebbe restituire $9$ (la somma di $4$ e $5$). Si scriva il file di header \texttt{sum.h} in cui si dichiara tale funzione.

\vspace*{1ex}
\begin{center}{\Large Esercizio 2} ($8$ punti)\end{center}
%
Si scriva un programma \texttt{main\_sum.c} che include la funzione dell'Esercizio~1 tramite
il file di header \texttt{sum.h}.
Il programma \texttt{main\_sum.c} deve contenere una funzione iniziale \texttt{main} che esegue
le seguenti operazioni:
\begin{enumerate}
\item legge da tastiera la lunghezza \texttt{length} di un array, richiedendola ad oltranza se fosse inserita negativa;
\item crea un array \texttt{elements} di \texttt{length} interi;
\item legge da tastiera gli elementi di tale array, uno alla volta;
\item chiama la funzione \texttt{sum} dell'Esercizio~1, passando \texttt{elements} e \texttt{length};
\item stampa sul video il risultato di tale chiamata.
\end{enumerate}

\vspace*{1ex}
\begin{center}{\Large Esercizio 3} ($10$ punti)\end{center}
%
Si definisca una struttura \texttt{trainingRecorder} che implementa un registro per memorizzare risultati di allenamenti podistici. Si scrivano i file \texttt{trainingRecorder.h} e \texttt{trainingRecorder.c} che implementano le funzioni:

\begin{itemize}
\item \texttt{struct trainingRecorder *constructTrainingRecorder(char *trainingName)}, che resti\-tui\-sce un nuovo registro vuoto con il nome indicato;

\item \texttt{void destructTrainingRecorder(struct trainingRecorder *this)}, che dealloca \texttt{this};

\item \texttt{void addRun(struct trainingRecorder *this, float length, float time)},
      che aggiunge un nuovo allenamento (\emph{run}) al registro memorizzandone la lunghezza del percorso (in chilometri) e il tempo impiegato (in minuti). Un numero illimitato di allenamenti possono essere aggiunti ad un registro;

\item \texttt{float averageSpeed(struct trainingRecorder *this)}, che restituisce la velocit\`a media degli allenamenti inseriti in \texttt{this}; se \texttt{this} non contenesse nessun allenamento, questa funzione deve restituire $0.0$;

\item \texttt{char *toString(struct trainingRecorder *this)}, che restituisce una nuova stringa
      ottenuta dal nome del registro \texttt{this} seguito dalla velocit\`a media e dal numero degli allenamenti in esso contenuti;

\item \texttt{int compareAverageSpeeds(struct trainingRecorder *tr1, struct trainingRecorder *tr2)}, che confronta \texttt{tr1} e \texttt{tr2} e restituisce -1 se la velocit\`a media del primo \`e minore di quella del secondo, 1 se la velocit\`a media del primo \`e maggiore di quella del secondo, 0 altrimenti. 
\end{itemize}
%
Se tutto \`e corretto, l'esecuzione del seguente programma \texttt{main\_trainingRecorder.c}:

\begin{lstlisting}
#include <stdlib.h>
#include <stdio.h>
#include "trainingRecorder.h"

int main(void) {
  struct trainingRecorder *tr1 = constructTrainingRecorder("New York Marathon");
  struct trainingRecorder *tr2 = constructTrainingRecorder("Rome Marathon");
  char *s;

  addRun(tr1, 10, 51);
  addRun(tr1, 15, 79);
  addRun(tr1, 20, 120);

  addRun(tr2, 8, 40);
  addRun(tr2, 35, 180);
  addRun(tr2, 20, 115);
  addRun(tr2, 40, 195);

  printf("%s\n", s = toString(tr1));
  free(s);
  printf("%s\n", s = toString(tr2));
  free(s);
  printf("Comparison:  %d\n", compareAverageSpeeds(tr1, tr2));

  destructTrainingRecorder(tr1); destructTrainingRecorder(tr2);
  return 0;
}
\end{lstlisting}

\noindent
deve stampare:

{\small
\begin{verbatim}
Training name: New York Marathon; average speed:  10.80 Km/h; number of runs: 3
Training name: Rome Marathon; average speed:  11.66 Km/h; number of runs: 4
Comparison:  -1
\end{verbatim}

\vspace*{1ex}
\begin{center}{\Large Esercizio 4} ($8$ punti)\end{center}

Si modifichi il programma \texttt{coppia.c} definendo le funzioni
\texttt{init}, \texttt{f}, \texttt{g} ed \texttt{h}, in modo che alla fine compili senza
errori e la sua esecuzione stampi:
%
\begin{verbatim}
Il valore della coppia e' uguale a (5,10)
Il valore della coppia e' uguale a (45,10)
Il valore della coppia e' uguale a (45,10)
Il valore della coppia e' uguale a (45,12)
\end{verbatim}
%
Non potete modificare altro in \texttt{coppia.c}; quindi, per esempio, non potete modificare
la funzione \texttt{main} o la \texttt{struct coppia}.

\end{document}
