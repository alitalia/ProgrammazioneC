\documentclass[12pt]{article}
\usepackage{amssymb}
\usepackage{amsmath}
\usepackage{color}
\usepackage{graphicx}

\definecolor{grey}{rgb}{0.3,0.3,0.3}

\usepackage{listings, framed}
\lstset{
  language=Java,
  showstringspaces=false,
  columns=flexible,
  basicstyle={\small\ttfamily},
  frame=none,
  numbers=none,
  keywordstyle=\bfseries\color{grey},
  commentstyle=\itshape\color{red},
  identifierstyle=\color{black},
  stringstyle=\color{blue},
  numberstyle={\ttfamily},
%  breaklines=true,
  breakatwhitespace=true,
  tabsize=3,
  escapechar=|
}

\thispagestyle{empty}
\setlength{\textwidth}{18.5cm}
\setlength{\topmargin}{-2.5cm}
\setlength{\textheight}{24.5cm}
\setlength{\oddsidemargin}{-1cm}
\setlength{\evensidemargin}{-1cm}
\begin{document}
\begin{center}{\LARGE Parziale di Programmazione I - BioInformatica}\\
\vspace*{-2ex}
\begin{center}
  \large 1 febbraio 2018 (tempo disponibile: 2 ore)
\end{center}
\end{center}

\begin{center}{\Large Esercizio 1} ($5$ punti)
\end{center}

Cosa stampa il seguente programma?

\begin{lstlisting}
#include <stdio.h>

int main(void) {
  int arr[] = { 22, 7, 7, 26, -4 };
  int i, sum = 0, counter = 0;

  for (i = 1; i < 5; i += 2, counter++)
    sum += arr[i];

  float result = sum / counter;
  printf("%.2f\n", result);
  return 0;
}
\end{lstlisting}

\vspace*{1ex}
\begin{center}{\Large Esercizio 2} ($9$ punti)
\end{center}
Si scriva una funzione \texttt{rotate\_right} che riceve un array di interi e la sua lunghezza e modifica l'array spostando i suoi elementi di una posizione a destra (cio\`e verso la fine dell'array); l'elemento che esce da destra deve rientrare da sinistra.

\vspace*{1ex}
\begin{center}{\Large Esercizio 3} ($13$ punti)\end{center}
%
Si completi il seguente programma scrivendo la funzione
\texttt{quasi\_max}, che riceve un array di interi e la sua lunghezza
e ritorna il quasi massimo dell'array, cio\`e il pi\'u grande numero
contenuto nell'array che non sia il massimo. Per esempio, sotto
dovr\`a venire stampato $23$, che \`e il quasi massimo di \texttt{arr}.
\textbf{Nota:} per semplicit\`a
si assuma che il quasi massimo dell'array
esista, quindi non ci si preoccupi di gestire i casi in cui
esso non esiste, per esempio quando l'array \`e vuoto
o ha un solo elemento. \textbf{Suggerimento:} definite funzioni
ausiliarie, se vi risultano di aiuto.
%
\begin{lstlisting}
#include <stdio.h>

int main(void) {
  int arr[] = { 28, 10, 7, 9, 14, 22, 23, 28, -4, 23 };
  printf("quasi massimo: %d\n", quasi_max(arr, 10));
  return 0;
}
\end{lstlisting}

\vspace*{1ex}
\begin{center}{\Large Esercizio 4} ($5$ punti)\end{center}

\begin{itemize}
\item Si pu\`o scrivere la funzione \texttt{quasi\_max} dell'esercizio precedente
  prima del \texttt{main}? Come?
\item Si pu\`o scriverla dopo il \texttt{main}? Come?
\item Si pu\`o scriverla in un altro file sorgente diverso da quello
  in cui sta il \texttt{main}? Come?
\end{itemize}

\end{document}
