\documentclass[12pt]{article}
\usepackage{amssymb}
\usepackage{amsmath}
\usepackage{graphicx}

\thispagestyle{empty}
\setlength{\textwidth}{18.5cm}
\setlength{\topmargin}{-2.5cm}
\setlength{\textheight}{24.5cm}
\setlength{\oddsidemargin}{-1cm}
\setlength{\evensidemargin}{-1cm}
\begin{document}
\begin{center}{\LARGE Esame di Programmazione I}\\
\vspace*{-2ex}
\begin{center}
  \large 2 febbraio 2015 (tempo disponibile: 2 ore)
\end{center}
\end{center}
%\mbox{}\\
%\vspace*{1ex}
\begin{center}{\Large Esercizio 1} ($11$ punti)
\end{center}
Si scriva una funzione \texttt{char *histo(int values[], int length)}
che restituisce una nuova stringa che rappresenta una barra orizzontale di 80 caratteri
divisa in segmenti di lunghezza proporzionale ai valori dell'array \texttt{values}.
Il paramentro \texttt{length} \`e la lunghezza di tale array. I segmenti sono fatti,
alternativamente, dai caratteri \texttt{*@+\#}.
Se tutto \`e corretto, l'esecuzione del seguente programma:

{\small
\begin{verbatim}
int main(void) {
  int values[] = { 8, 12, 0, 3, 5 };
  char *s = histo(values, 5);
  printf("%s", s);
  free(s);
  return 0;
}
\end{verbatim}}

\noindent
dovr\`a stampare:
%
{\small
\begin{verbatim}
***********************@@@@@@@@@@@@@@@@@@@@@@@@@@@@@@@@@@@++++++++##############
\end{verbatim}}

\vspace*{1ex}
\begin{center}{\Large Esercizio 2} ($11$ punti)\end{center}
%
Si considerino le liste di \texttt{char} come viste a lezione. Si definisca una funzione
\textbf{ricorsiva}
%
\begin{verbatim}
int *frequency(struct list *this)
\end{verbatim}
%
che restituisce un nuovo array di interi contenente
le frequenze di ciascun carattere nella lista
\texttt{this}. Questo significa che il primo elemento dell'array sar\`a il numero di
caratteri \texttt{'a'} che si trovano in \texttt{this}, il secondo elemento sar\`a il
numero di caratteri \texttt{'b'} che si trovano in \texttt{this} e cos\`{\i} via. Per
esempio, l'esecuzione del seguente programma:

{\small
\begin{verbatim}
int main(void) {
  struct list *l = construct_list
    ('f', construct_list('a', construct_list('f', construct_list ('z', construct_list
        ('a', construct_list('f', construct_list('k', construct_list('m', NULL))))))));
  printf("l = "); print_list(l); printf("\n");
  int *freq = frequency(l);
  int pos;
  for (pos = 0; pos < 26; pos++)
    printf("%i ", freq[pos]);
  printf("\n");
  free(freq);
  return 0;
}
\end{verbatim}}

\noindent
dovr\`a stampare
%
{\small
\begin{verbatim}
l = [f, a, f, z, a, f, k, m]
2 0 0 0 0 3 0 0 0 0 1 0 1 0 0 0 0 0 0 0 0 0 0 0 0 1
\end{verbatim}}

\vspace*{1ex}
\begin{center}{\Large Esercizio 3} ($9$ punti)\end{center}
%
Si definisca una struttura \texttt{banca} che implementa una banca con massimo 10 correntisti,
identificati per nome.
Si scrivano i file \texttt{banca.h} e \texttt{banca.c} implementando le funzioni:
%
\begin{itemize}
\item \texttt{struct banca *construct\_banca()} che restituisce una nuova banca,
      ancora senza correntisti;
\item \texttt{void destruct\_banca(struct banca *this)} che dealloca la banca \texttt{this};
\item \texttt{void deposita(struct banca *this, char *nome, double soldi)},
      che aggiunge i \texttt{soldi} indicati sul conto del correntista chiamato \texttt{nome} che si
      trova nella banca \texttt{this}. Se il correntista non \`e ancora presente
      nella banca e non si
      \`e ancora arrivati al massimo di 10 correntisti, esso viene aggiunto come nuovo correntista con deposito
      iniziale pari a \texttt{soldi}.  Se il correntista non \`e ancora presente
      nella banca e si \`e gi\`a arrivati a 10 correntisti, questa funzione non
      fa nulla;
\item \texttt{void interessi(struct banca *this, double percent)}, che aggiungi
      a tutti i correntisti un interesse pari al \texttt{percent} per cento;
\item \texttt{char *toString(struct banca *this)}, che restituisce una nuova stringa del tipo
      \vspace{-1ex}
      \begin{verbatim}
        Fausto: 113.840000
        Samantha: 1023.030000
        Alessandra: 11.340000
      \end{verbatim}
      \vspace{-3ex}
      in cui cio\`e sono riportati i nomi dei correntisti e l'entit\`a del deposito che hanno presso
      la banca.
\end{itemize}
%
Se tutto \`e corretto, l'esecuzione del seguente programma:

{\small
\begin{verbatim}
#include <stdlib.h>
#include <stdio.h>
#include "banca.h"

int main(void) {
  struct banca *b = construct_banca();
  char *s;
  deposita(b, "Fausto", 112.14); deposita(b, "Samantha", 1023.03);
  deposita(b, "Fausto", 13.00);  deposita(b, "Alessandra", 11.34);
  printf("%s\n", s = toString(b));
  free(s);
  interesse(b, 2.50);
  printf("%s\n", s = toString(b)); free(s);
  destruct_banca(b);
  return 0;
}
\end{verbatim}}

\noindent
dovr\`a stampare:

{\small
\begin{verbatim}
Fausto: 125.140000
Samantha: 1023.030000
Alessandra: 11.340000

Fausto: 128.268500
Samantha: 1048.605750
Alessandra: 11.623500
\end{verbatim}}

\end{document}
