\documentclass{article}[10pt]
\usepackage[pdftex]{graphicx}
\usepackage{amsfonts}
\usepackage[italian]{babel}
%****************enlarge layout
\textheight     243.5mm
\topmargin      -20.0mm
\textwidth      480pt
\hoffset        -80pt
%*****************theorems and such
\newcounter{esnu}
\newenvironment{esercizio}{\medskip \noindent {\bf Esercizio\addtocounter{esnu}{1} \arabic{esnu}}}{}
\pagestyle{empty}
\newcommand{\liff}{\mathrel{\leftrightarrow}}   % Logical IFF Symbol
\newcommand{\metaiff}{\Longleftrightarrow}      %iff in metatheory

\begin{document}

\begin{tabular}{llclcr}
 \hspace{-35pt} &{\bf COGNOME:} & \hspace{100pt}        &{\bf NOME:}    & \hspace{100pt}        &{\bf MATRICOLA:}\hspace{35pt} \\
\hline
\end{tabular}
\begin{center} {\bf Esame di Programmazione I, 11 luglio 2012. 2 ore}\end{center}

\begin{esercizio}
\textbf{[12 punti]}
Si scriva una funzione

\begin{verbatim}
char *replace(const char *s, const char *old, const char *new)
\end{verbatim}

\noindent
che restituisce una nuova stringa ottenuta sostituendo
con \texttt{new} la prima occorrenza (quella pi\`u a sinistra) di \texttt{old} dentro \texttt{s}.
Se tutto \`e corretto, l'esecuzione del seguente programma:

{\small
\begin{verbatim}
#include <stdio.h>

int main(void) {
  const char *x = "ciao, come stai?";
  char *r = replace(x, "come stai", "how are you");
  printf("%s\n", r);
  free(r);
  r = replace(x, "a", "e");
  printf("%s\n", r);
  free(r);
  r = replace(x, "", "_inizio_");
  printf("%s\n", r);
  free(r);
  r = replace(x, "hello", "ciao");
  printf("%s\n", r);
  free(r);
  return 0;
}
\end{verbatim}
}

\noindent
deve stampare:

{\small
\begin{verbatim}
ciao, how are you?
cieo, come stai?
_inizio_ciao, come stai?
ciao, come stai?
\end{verbatim}
}

\end{esercizio}

\begin{esercizio}
\textbf{[9 punti]}
Si considerino le liste viste a lezione. Si implementi una funzione ricorsiva:
\begin{verbatim}
int prefix(struct list *this, struct list *other);
\end{verbatim}
che restituisce \emph{vero} se e solo se \texttt{other} \`e un prefisso di \texttt{this} (cio\`e se e solo se \texttt{this} comincia con \texttt{other}). Si noti che la lista
vuota \`e un prefisso di qualsiasi altra lista.

Se tutto \`e corretto, l'esecuzione del programma:

{\small
\begin{verbatim}
#include <stdio.h>
#include <stdlib.h>
#include "list.h"

int main(void) {
  struct list *l = construct_list(11, construct_list(13, construct_list
      (-4, construct_list(13, construct_list(-6, construct_list(0, construct_list(1, NULL)))))));

  struct list *p1 = construct_list(11, construct_list(13, NULL));
  struct list *p2 = construct_list(13, construct_list(-4, construct_list(13, NULL)));

  print_list(p1);
  if (!prefix(l, p1))
    printf(" non");
  printf(" e' un prefisso di ");
  print_list(l);
  printf("\n");

  print_list(p2);
  if (!prefix(l, p2))
    printf(" non");
  printf(" e' un prefisso di ");
  print_list(l);
  printf("\n");

  return 0;
}
\end{verbatim}
}

\noindent
dovr\`a stampare:

{\small
\begin{verbatim}
[11, 13] e' un prefisso di [11, 13, -4, 13, -6, 0, 1]
[13, -4, 13] non e' un prefisso di [11, 13, -4, 13, -6, 0, 1]
\end{verbatim}
}
%
\end{esercizio}
%
\begin{esercizio}
\textbf{[11 punti]}
%
Si definisca una struttura \texttt{radio} che implementa una radio con 10 canali
(numerati da 0 a 9). Si scrivano i file \texttt{radio.h} e \texttt{radio.c} implementando
le funzioni:
%
\begin{itemize}
\item \texttt{struct radio *construct\_radio()} che restituisce una nuova radio,
      i cui dieci canali non sono ancora memorizzati e che \`e sintonizzata sul canale 0;
\item \texttt{void destruct\_radio(struct radio *this)} che dealloca la radio \texttt{this};
\item \texttt{void memorizza\_canale(struct radio *this, int numero, char *nome, float frequenza)},
      che\linebreak memorizza il canale di numero indicato di \texttt{this} sull'emittente di nome
      \texttt{nome} la cui frequenza \`e \texttt{frequenza};
\item \texttt{void sintonizza\_su(struct radio *this, int numero)}, che sintonizza la radio
      \texttt{this} sul canale di numero indicato;
\item \texttt{char *toString(struct radio *this)}, che restituisce una nuova stringa del
      tipo
      \begin{verbatim}
        Radio Arena: frequenza 102.5
      \end{verbatim}
      (il nome dell'emittente seguito dall'indicazione
      della frequenza). Se la radio \`e sintonizzata su un canale non ancora memorizzato,
      questa funzione deve restituire la nuova stringa \texttt{canale non memorizzato}.
\end{itemize}
%
Se tutto \`e corretto, l'esecuzione del seguente programma:

{\small
\begin{verbatim}
#include <stdlib.h>
#include <stdio.h>
#include "radio.h"

int main(void) {
  struct radio *r = construct_radio();
  char *s = toString(r);
  printf("%s\n", s);
  free(s);
  memorizza_canale(r, 3, "Radio Patchanka", 98.1);
  memorizza_canale(r, 0, "Radio Castelrotto", 100.5);
  memorizza_canale(r, 9, "Radiofutura", 105.6);
  sintonizza_su(r, 3);
  printf("%s\n", s = toString(r));
  free(s);
  memorizza_canale(r, 0, "Radio Arena", 102.5);
  sintonizza_su(r, 0);
  printf("%s\n", s = toString(r));
  free(s);
  destruct_radio(r);
  return 0;
}
\end{verbatim}}

\noindent
deve stampare:

{\small
\begin{verbatim}
canale non memorizzato
Radio Patchanka: frequenza 98.1
Radio Arena: frequenza 102.5
\end{verbatim}}

\end{esercizio}
%
\end{document}
