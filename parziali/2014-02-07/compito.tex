\documentclass[italian,12pt]{article}
\usepackage{babel}
\usepackage{amssymb}
\usepackage{amsmath}
\usepackage{graphicx}

\thispagestyle{empty}
\setlength{\textwidth}{18.5cm}
\setlength{\topmargin}{-2.5cm}
\setlength{\textheight}{24.5cm}
\setlength{\oddsidemargin}{-1cm}
\setlength{\evensidemargin}{-1cm}
\begin{document}
\begin{center}{\LARGE Prima prova parziale di Programmazione I}\\
\begin{center}
  \Large 7 febbraio 2014 (tempo disponibile: 2 ore)
\end{center}
\end{center}
%\mbox{}\\
\begin{center}{\Large Esercizio 1}\\
($10$ punti)
\end{center}
Si scriva una funzione
\begin{verbatim}
  double e(int precision)
\end{verbatim}
che restituisce un'approssimazione della costante di Eulero $e$ calcolata come:
\[
  \sum\limits_{n=0}^{\mathtt{precision}}\frac{1}{n!}
\]

\noindent
Si scriva quindi un \texttt{main} che chiede all'utente \texttt{precision} e stampa l'approssimazione
di $e$ calcolata con la precedente funzione e la precisione inserita, usando 20 cifre
decimali. Per esempio, una possibile esecuzione di tale programma potrebbe essere:
%
{\small
\begin{verbatim}
$ ./a.out
precision: 7
2.7182539682539683668
\end{verbatim}
}
%
\begin{center}{\Large Esercizio 2}\\
($10$ punti)
\end{center}
%
Si scriva una funzione
%
\begin{verbatim}
  void swap(int array[], int length)
\end{verbatim}
%
che riceve come parametri un array di lunghezza
\texttt{length}, considera l'array come una sequenza di
triplette di interi e, per ogni tripletta, inverte il primo e il terzo
elemento della tripletta. Se alla fine rimangono 1 o 2 elementi, non vengono
modificati.

Se tutto \`e corretto, l'esecuzione del seguente \texttt{main}:
{\small
\begin{verbatim}
int main(void) {
  int arr[] = { 13, 0, 34, -5, -6, 10, 34, -28, 44, 71, 9 };
  int pos;

  swap(arr, 11);
  for (pos = 0; pos < 11; pos++)
    printf("%i ", arr[pos]);
  printf("\n");
  return 0;
}

\end{verbatim}
}

\noindent
stamper\`a
%
{\small
\begin{verbatim}
34 0 13 10 -6 -5 44 -28 34 71 9
\end{verbatim}
}
%
\begin{center}{\Large Esercizio 3}\\
($12$ punti)\end{center}
%
Si scriva un programma \texttt{primes.c} che definisce la funzione \textbf{ricorsiva}
\begin{verbatim}
  int count_primes(int max)
\end{verbatim}
%
che restituisce la quantit\`a di numeri primi tra 0 e \texttt{max} inclusi.
Si scriva quindi un \texttt{main} che legge \texttt{max} da tastiera
e stampa quanti numeri primi ci sono tra 0 e \texttt{max} inclusi.
Per esempio, una possibile esecuzione del programma potrebbe essere:
%
{\small
\begin{verbatim}
$ ./a.out
max: 13
Ci sono 6 numeri primi tra 0 e 13
\end{verbatim}
}

\end{document}
